% Prof. Dr. Ausberto S. Castro Vera
% UENF - CCT - LCMAT - Curso de Ci\^{e}ncia da Computa\c{c}\~{a}o
% Campos, RJ,  2023
% Disciplina: Paradigmas de Linguagens de Programa\c{c}\~{a}o
%



\chapter{Considera\c{c}\~{o}es Finais}


Através de modelos teóricos e práticos amplamente difundidos, buscou-se elucidar os primeiros passos na trilha dessa linguagem de programação tão versátil. Este livro tem como objetivo englobar desde programadores que estão começando a se aventurar nessa área, passando por desenvolvedores experientes em busca de tópicos específicos, até matemáticos, estatísticos ou cientistas de dados que necessitam de uma nova tecnologia para otimizar seu trabalho. Assim, o livro se propõe a ser um convite para que mais pessoas se juntem à vasta e vibrante comunidade do R.\par

Esta jornada pelo conhecimento se deu através de extensa pesquisa no âmbito científico e acadêmico, utilizando-se também de materiais amplamente referenciados nos capítulos, cujas referências bibliográficas podem ser encontradas. Durante essa jornada, surgiram desafios que apenas contribuíram para o crescimento do autor, visto que foi necessário criar códigos específicos para a construção das seções.\par

Como considerações finais e como recomendação para aprofundar o conhecimento na linguagem, é recomendado, primeiramente, explorar cada um dos livros mencionados nos exemplos. Além disso, é indicada uma pesquisa mais detalhada sobre os diversos pacotes disponíveis na comunidade do R. Com suas diversas funções e capacidades, esses pacotes são o que tornam essa linguagem tão útil. Vale a pena dedicar esforço para aprender o máximo possível sobre suas utilidades, para assim ter acesso a ferramentas cada vez mais poderosas.\par



   \begin{figure}[H]
    \begin{center}
        \caption{Aplica\c{c}\~{a}o da Linguagem R} \label{ling2}
        \includegraphics[width=12cm]{R02.png} \\
        {\tiny \sf Fonte: O autor }
    \end{center}
   \end{figure} 